\section{Background}

\begin{frame}{Background}
    \framesubtitle{Transformer e Self-Attention}
    \begin{itemize}
        \item \textbf{Transformer} (Vaswani et al., 2017): Architettura basata su Self-Attention.
        \item Elaborazione parallela dell'intera sequenza.
        \item \textbf{Self-Attention}:
        \[ \text{Attention}(Q, K, V) = \text{softmax}\left(\frac{QK^T}{\sqrt{d_k}}\right)V \]
        \item Permette di catturare relazioni a lungo raggio.
    \end{itemize}
\end{frame}

\begin{frame}{Background}
    \framesubtitle{Large Language Models (LLM)}
    \begin{itemize}
        \item Modelli probabilistici addestrati su enormi corpora (Next Token Prediction).
        \item \textbf{Scaling Laws}: Prestazioni migliorano con parametri e dati.
        \item Ciclo di vita:
        \begin{enumerate}
            \item Pre-training (Unsupervised)
            \item Supervised Fine-Tuning (SFT)
            \item Alignment (RLHF/DPO)
        \end{enumerate}
    \end{itemize}
\end{frame}

\begin{frame}{Background}
    \framesubtitle{Allucinazioni}
    \begin{itemize}
        \item \textbf{Tipologie di allucinazione}:
        \begin{itemize}
            \item Fattuali (Falsità oggettive)
            \item Logiche (Incoerenze nel ragionamento)
            \item Incoerenza di Contesto (Contraddizione del prompt)
        \end{itemize}
        \item \textbf{Cause}: Compressione con perdita, dati rumorosi, Sycophancy.
    \end{itemize}
\end{frame}

