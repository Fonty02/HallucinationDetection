\section{Metodologia}

\begin{frame}{Metodologia}
    \framesubtitle{Dataset Utilizzati}
    \begin{itemize}
        \item \textbf{Belief Bank Facts}:
        \begin{itemize}
            \item Fatti affermativi e negati (es. "An eagle is a bird").
            \item 27.416 affermazioni (Bilanciato).
            \item Rilevamento \textit{Factual Hallucinations}.
        \end{itemize}
        \item \textbf{Belief Bank Constraints}:
        \begin{itemize}
            \item Implicazioni e mutue esclusioni.
            \item 25.756 affermazioni.
            \item Rilevamento \textit{Logical Inconsistencies}.
        \end{itemize}
        \item \textbf{HaluEval}:
        \begin{itemize}
            \item Contesti conversazionali complessi.
            \item 10.000 esempi.
        \end{itemize}
    \end{itemize}
\end{frame}



\begin{frame}{Metodologia}
    \textbf{Pipeline di Estrazione}
    \begin{enumerate}
        \item Prompting del modello (es. "Is the fact true?").
        \item Estrazione attivazioni da tutti i layer e componenti (Attention, MLP, Hidden).
        \item Etichettatura basata sulla risposta generata (Yes/No).
    \end{enumerate}

    \textbf{Pipeline generale di addestramento}

    \begin{enumerate}
    \item Addestramento prober sulle attivazioni del modello trainer
    \item Allineamento delle attivazioni in comune tra trainer e tester
    \item Valutazione delle attivazioni allineate del tester sul prober del trainer
    \end{enumerate}
\end{frame}

\begin{frame}{Metodologia: Baseline}
    \framesubtitle{Approccio Lineare}
    \begin{figure}
        \centering
        \includegraphics[width=0.9\textwidth, height=0.7\textheight]{images/modelsss/AppL.png}
        \caption{Pipeline Baseline: Logistic Regression su Trainer, Ridge Regression per allineamento Tester.}
    \end{figure}
\end{frame}

\begin{frame}{Metodologia: Approccio Ibrido}
    \framesubtitle{Allineamento Non-Lineare + Prober Lineare}
    \begin{figure}
        \centering
        \includegraphics[width=0.9\textwidth, height=0.6\textheight]{images/modelsss/AppH.png}
        \caption{Pipeline Ibrida: AlignmentNetwork non-lineare per proiettare il Tester, classificatore lineare fisso.}
    \end{figure}
\end{frame}

\begin{frame}{Metodologia: Approccio Non-Lineare Completo}
    \framesubtitle{Allineamento Non-Lineare + Prober Non-Lineare}
    \begin{figure}
        \centering
        \includegraphics[width=0.9\textwidth, height=0.6\textheight]{images/modelsss/PipeApp1.png}
        \caption{Pipeline Completa: AlignmentNetwork e MLP Prober entrambi non-lineari.}
    \end{figure}
\end{frame}

\begin{frame}{Metodologia: Approccio Non-Lineare Ridotto}
    \framesubtitle{Riduzione Dimensionale con Autoencoder}
    \begin{figure}
        \centering
        \includegraphics[width=0.9\textwidth, height=0.7\textheight]{images/modelsss/PipeApp2.png}
        \caption{Pipeline Ridotta: Autoencoder per ridurre il rumore, poi allineamento nello spazio latente.}
    \end{figure}
\end{frame}

\begin{frame}{Metodologia: One-For-All}
    \framesubtitle{Frozen Head \& Encoder Adaptation}
    \begin{figure}
        \centering
        \includegraphics[width=0.9\textwidth, height=0.7\textheight]{images/modelsss/PipeApp3.png}
        \caption{Pipeline One-For-All: Encoder specifico per modello, Classification Head congelata dal Trainer.}
    \end{figure}
\end{frame}

