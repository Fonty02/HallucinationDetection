\section{Discussione}

\begin{frame}{Discussione}
    \framesubtitle{Confronto tra Metodi}
    \begin{itemize}
        \item I metodi di trasferibilità sono generalmente efficaci.
        \item \textbf{One-For-All} emerge come il metodo più promettente e robusto.
        \item La trasferibilità è particolarmente alta per task di verifica fattuale semplice.
    \end{itemize}
\end{frame}

\begin{frame}{Discussione}
    \framesubtitle{Analisi dei Dataset}
    \begin{itemize}
        \item \textbf{BeliefBank (Facts \& Calibration)}:
        \begin{itemize}
            \item Alta separabilità lineare (Acc > 95\%).
            \item Concetto di verità "universale" e ben definito.
        \end{itemize}
        \item \textbf{HaluEval}:
        \begin{itemize}
            \item Molto più complesso (Acc 70-80\%).
            \item Le allucinazioni sono più specifiche del modello e del contesto.
            \item Minore trasferibilità.
        \end{itemize}
    \end{itemize}
\end{frame}

\begin{frame}{Discussione}
    \framesubtitle{Analisi dei Modelli e Layer}
    \begin{itemize}
        \item \textbf{Falcon3 \& Qwen2.5}: Altissima compatibilità, suggerisce similarità strutturale.
        \item \textbf{Llama-3.1 \& Gemma-2}: Buona compatibilità, ma minore su task complessi.
        \item \textbf{Layer}:
        \begin{itemize}
            \item \textit{Attention} e \textit{Hidden States} spesso più stabili di \textit{MLP}.
            \item One-For-All è robusto su tutte le componenti.
        \end{itemize}
    \end{itemize}
\end{frame}

\begin{frame}{Discussione}
    \framesubtitle{Asimmetria Cross-Domain}
    \begin{itemize}
        \item \textbf{Simple $\rightarrow$ Complex}: Funziona molto bene.
        \item \textbf{Complex $\rightarrow$ Simple}: Funziona peggio.
        \item \textbf{Ipotesi}: Le strutture latenti apprese su fatti semplici sono fondamentali e universali. Quelle apprese su contesti complessi sono più rumorose o specifiche.
    \end{itemize}
\end{frame}